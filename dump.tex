“The first principle is that you must not fool yourself and you are the easiest person to fool.”
― Richard P. Feynman

%\mbox{}\\\noindent\emph{Printing should be invisible.} --- Beatrice Warde % TO DO: clarification in text body
%\mbox{}\\\noindent\emph{By the way, Kitty, if only you'd been really with me in my dream, there was one thing you would have enjoyed——I had such a quantity of poetry said to me, all about fishes!} --- Through the Looking-Glass, Beatrice Warde

%\mbox{}\\\noindent\emph{One man's transparency is another's humiliation.} --- Gerry Adams % AN ALTERNATIVE

%\mbox{}\\\noindent\emph{The type that, through any arbitrary warping of design or excess of ``color," get in the way of the mental picture to be conveyed, is a bad type.} --- Beatrice Warde
% "get in the way" is the "get" [sic]?



% Feedback from AR to JB:
% - Don't \cite twice, use author's name.
% - Try to rephrase unless direct citation is important.
% - Reduce commas where possible.

% ========================================
% ^^^ above here is actual text
% ========================================
==> DIVIDER DIVIDER DIVIDER <==
% ========================================
% vvv below here are just notes
% ========================================

Hypothesis 1: The green background makes participants more likely to fall for a phishing attack. That is, we expected more positive responses in the ``fraud with green bar'' group compared with the ``fraud without green bar'' group.

Hypothesis 2: compare fraud with green bar to authentic with ev-ssl

Hypothesis 3: compare authentic with ev-ssl with fraud without green bar



We are interested in comparing the effectiveness of the phishing attacks. Call $X_{green}^{(j)} = $ ``the $j$-th participant who was shown a green-bar screenshot has answered that they trust the website'' and similarly for $X_{t}^{(j)}$ with $t$ from ``authentic'', ``green'' and ``plain.'' Define $D_{green}^{(j)} = $ ``the $j$-th participant was shown a green-bar screenshot'' and $D_{t}^{(j)}$, then $D_{t}^{(j)}\stackrel{i.i.d.}{\sim}B(1,{1\over 3})$ for each $t$. We reasonably assume that $X_{t}^{(j)}\stackrel{i.i.d.}{\sim}B(1,p_{t})$ for each $t$.

The relative effectiveness of each attack is covered by six trivial hypotheses: $H_{1,t,t',<} := \{p_{t} < p_{t'}\}, t\neq t'$. The respective null hypotheses are then: $H_{0,t,t',\ge} := \{p_{t}\ge p_{t'}\}$. So $H_{1,t,t',<}$ means intuitively ``the screenshot type $t'$ is more trustworthy than the type $t$'' and when $t'$ is an attack ``the attack $t'$ is more effective than $t$'' resp. ``the attack $t'$ is unrecognized'' when $t = $ ``authentic''. $t'$ can also be ``authentic'', in which case ``the attack $t$ is detected by users.''

We test 6 hypotheses for .....
